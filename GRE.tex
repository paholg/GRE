\documentclass{article}
\usepackage{amsmath}
\usepackage[margin=1in]{geometry} %added margin control. feel free to change back to how it was by commenting this line.
\setlength\parindent{0pt}

\begin{document}
\title{GRE Equation Sheet}
\author{Mattson Thieme \and Rene Zeto \and Steve Ellefson}
\maketitle



\section{Dimensions}
Energy\begin{equation}\frac{kg*m^2}{s^2}
\end{equation} 
Momentum\begin{equation}\frac{kg*m}{s}\end{equation}






\section{Classical Mechanics}
3D Wave Equation\begin{equation}\frac{1}{v^2} \frac{\partial^2\psi}{\partial t}= \nabla^2 \psi \end{equation}

\begin{equation}v=\sqrt{\frac{\tau}{\mu}},  \tau=tension, \mu=mass density \end{equation}

Traveling Waves\begin{equation}A\sin(kx+\omega t + \phi_0)+B\sin(kx+\omega t +\phi_0)\end{equation}

Standing Waves\begin{equation}A\sin(kx)\sin(\omega t+\phi_0)+B\cos(kx)\cos(\omega t+\phi_0)\end{equation}

Quality factor\begin{equation}Q=\frac{\omega}{2\beta}=2\pi \frac{energy-lost}{energy-stored}\end{equation}
$\Rightarrow$where $\beta$ is the half width max height? Half height max width? words

Beat Frequency\begin{equation}f_B=|f_1-f_2|.....T_B = \frac{1}{f_B}\end{equation}

Decibels(measured as Intensity {\bf above} threshold of human hearing at 1000 Hz)\begin{equation}I(dB)=10log_{10}\frac{I}{I_0}\end{equation}
$\Rightarrow$Where $I_0 = 10^{-12} \frac{watts}{m^2}$

Elastic Collisions\begin{equation}V_{f1}=V_{i1}(\frac{m_1-m_2}{m_1+m_2})+V_{i2}(\frac{2m_2}{m_1+m_2})\end{equation}
\begin{equation}V_{f2}=V_{i1}(\frac{2m_1}{m_1+m_2})+V_{i2}(\frac{m_2-m_1}{m_1+m_2})\end{equation}

Parallel Axis Theorem
\begin{equation}
  I_p = I_{cm} + Mh^2
\end{equation}
Where M is the mass of the object (?), and h is the distance between the center of mass and the pivot point p. 

Impulse\begin{equation}I = \int_{t_i}^{t_f} Fdt = \Delta p = J \end{equation}






\section{Quantum}









\section{E/M}
Maxwell's Equations in Differential form 
%mattson: there's some useful formatting stuff for vectors in latex
%incase you didn't know about them. here's an example:

%% \begin{equation}
%%   \vec{\nabla} \cdot \vec{E} = \frac{\rho(\vec{r})}{\epsilon_0}
%% \end{equation}

\begin{equation}\nabla \circ E=\frac{\rho}{\epsilon_0}\end{equation}
\begin{equation}\nabla \times B=\mu_0 J + \frac{dE}{dt}\end{equation}
\begin{equation}\nabla \times E=-\frac{dB}{dt}\end{equation}
\begin{equation}\nabla \circ B=0\end{equation}
$\Rightarrow$ Equations 3 and 4 are ``Sourcefull'' while 5 and 6 are ``Source free''
Maxwell's Equations in Integral form
\begin{equation}\int E \circ dA=\frac{q_{enc}}{\epsilon}\end{equation}
\begin{equation}\int B \circ dA=0\end{equation}
\begin{equation}\int E \circ dr=-\frac{\partial}{\partial t}\int B \circ dA \end{equation}
\begin{equation}\int B \circ dr=\mu_0 \int J \circ dA + \epsilon_0 \int \frac{dE}{dt}\circ dA\end{equation}

Force on Charge particles\begin{equation}F=q(E+V\times B)\end{equation}

Force density\begin{equation}F=\rho E + J\times V\end{equation}

Field Energy density\begin{equation}=\frac{1}{2} (\epsilon_0 E^2 + \frac{1}{\mu_0 B^2})\end{equation}

$\vec{P}$ density \begin{equation}\epsilon_0 E \times B = \epsilon_0 \mu_0 \vec{S}\end{equation}

$\vec{L}$ density\begin{equation}\vec{r}\times (\epsilon_0 E \times B)\end{equation}

E field from point charge\begin{equation}E=\frac{kq}{(r-r_0)^2}\end{equation}

Potential from point charge\begin{equation}V=\frac{kq}{r-r_0}\end{equation}










\section{Thermal Physics}









\section{Special Relativity}









\section{Oscillations}
Energy of an Oscillator\begin{equation}T+U=\frac{1}{2}mv^2 + \frac{1}{2}kx^2\end{equation}
Physical Pendulum\begin{equation}\omega = \sqrt{\frac{g}{L}}.....T=2\pi \sqrt{\frac{L}{g}}\end{equation}

Vibrations on a String:

Fixed-Fixed\begin{equation}A\sin(kx)\cos(\omega t + \phi_0), k = \frac{n\pi}{L}\end{equation}

Fixed-Free\begin{equation}A\sin(kx)\cos(\omega t+\phi_0), k = \frac{(2n+1)\pi}{L}\end{equation}

Free-Free\begin{equation}A\sin(kx)\cos(\omega t + \phi_0), k= \frac{n\pi}{L}\end{equation}


\section{Wave Optics}








\section{Ray Optics}









\section{Laboratory and Electronics}
Polarized Light through Polarizer\begin{equation}I_{transmitted} = I_0\cos^2(\theta)\end{equation}

Non-Polarized Light through Polarizer\begin{equation}I_{transmitted}=\frac{1}{2}I_0\end{equation}

$\Rightarrow$ I still haven't the foggiest idea what this means








\section{Condensed Matter}









\section{Nuclear and Particle Physics}









\section{Mathematics and Statistics}
{\bf Noethers Theorem:} If a system has a continuous symmetry property, then there are corresponding quantities whose values are conserved in time.  These are as follows:

\begin{tabular}{ l | r}
  Symmetry & Invariant\\
  Time & Energy\\
  Translation & Linear Momentum\\
  Rotation & Angular Momentum\\
  Gauge? & Charge\\
  Reflection & Parity\\
\end{tabular}

${\bf General Fourier}$ \begin{equation}a_n=\int_0^T sin(\frac{n\pi}{L}x)f(x)dx\end{equation}
\begin{equation}f(x)=\sum_0^{\infty} a_n sin(n x)\end{equation}




\section{Cosmology}









\section{Fluids}









\section{Constants}
Boltzmann's constant k=$1.38\times10^{-23} J/K$

e=$1.602\times10^{-11}C$

G=$6.672\times10^{-11}Nm^2/kg^2$

h=$6.626\times10^{-34}Js$

$N_A=6.022\times10^{23}molecules/mol$

Gas Constant(R)=8.3145 J/mol K

$m_e=9.109\times10^{-31}kg$

$m_n=1.672\times10^{-27}kg$

$m_p=1.672\times10^{-27}kg$

Permitivity of free space $\epsilon_0=8.854\times10^{-12}C^2/N m^2$

$\Rightarrow$ k=$\frac{1}{4\pi \epsilon_0}=8.987\times10^{9} Nm^2/C^2$

Permeability of free space $\mu_0=4\pi \times10^{-7}Nm^2/A m$

$\frac{1}{\epsilon_0 \mu_0} = c^2$









\section{Additional Questions/Review}









\end{document}
