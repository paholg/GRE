\documentclass{article}
\usepackage{amsmath}
\usepackage[margin=.5in]{geometry}
\setlength\parindent{0pt}

\begin{document}
\title{GRE Equation Sheet}
\author{Mattson Thieme \and Rene Zeto \and Steve Ellefson}
\maketitle


\section{Dimensions} Some of these are in SI units, more intuitive.\\
\\
Energy\begin{equation}\frac{kg*m^2}{s^2}
\end{equation}
Momentum\begin{equation}\frac{kg*m}{s}\end{equation}
Volts\begin{equation}\frac{J}{c}\end{equation}
Current\begin{equation}\frac{c}{s}\end{equation}
Z from Ohm's law for AC current\begin{equation}\frac{Js}{c^2}\end{equation}
Power\begin{equation}\frac{J}{s}\end{equation}






\section{Classical Mechanics}
{\bf{3D Wave Equation}}\begin{equation}\frac{1}{v^2} \frac{\partial^2\psi}{\partial t}= \nabla^2 \psi \end{equation}

\begin{equation}v=\sqrt{\frac{\tau}{\mu}},  \tau=tension, \mu=mass density \end{equation}

{\bf{Traveling Waves}}\begin{equation}A\sin(kx+\omega t + \phi_0)+B\sin(kx+\omega t +\phi_0)\end{equation}

\hspace*{.5in}Amplitude = A\\
\hspace*{.5in}Velocity\begin{equation}\vec{V}=\frac{\omega}{k}\end{equation}
\hspace*{.5in}Angular Frequency\begin{equation}\omega = 2\pi f \end{equation}
\hspace*{.5in}Wave vector\begin{equation}k = \frac{2\pi}{\lambda}\end{equation}
\hspace*{.5in}Period\begin{equation} T = \frac{2\pi}{\omega}\end{equation}

{{\bf{Standing Waves}}\begin{equation}A\sin(kx)\sin(\omega t+\phi_0)+B\cos(kx)\cos(\omega t+\phi_0)\end{equation}

{\bf{Quality factor}}\begin{equation}Q=\frac{\omega}{2\beta}=2\pi \frac{energy-lost}{energy-stored}\end{equation}
$\hspace*{.5in}$where $\beta$ is the half width max height? Half height max width? words
\\
\\
{\bf{Beat Frequency}}\begin{equation}f_B=|f_1-f_2|.....T_B = \frac{1}{f_B}\end{equation}

{\bf{Decibels}}(measured as Intensity above threshold of human hearing at 1000 Hz)\begin{equation}I(dB)=10log_{10}\frac{I}{I_0}\end{equation}
$\Rightarrow$Where $I_0 = 10^{-12} \frac{watts}{m^2}$\\
\\
{\bf{Sonic Boom}}\begin{equation}\sin{\theta} = \frac{s}{v}\end{equation}
$\Rightarrow$ where s is the speed of sound, v speed of source.  **Note: the right angle when drawing this diagram is with respect to the wave front, not the path of the source.\\
\\
{\bf{Elastic Collisions}}\begin{equation}V_{f1}=V_{i1}(\frac{m_1-m_2}{m_1+m_2})+V_{i2}(\frac{2m_2}{m_1+m_2})\end{equation}
\begin{equation}V_{f2}=V_{i1}(\frac{2m_1}{m_1+m_2})+V_{i2}(\frac{m_2-m_1}{m_1+m_2})\end{equation}

{\bf{Parallel Axis Theorem}}
\begin{equation}
  I_p = I_{cm} + Mh^2
\end{equation}
$\hspace*{.5in}$Where M is the mass of the object (?), and h is the distance between the center of mass and the pivot point p. 

\subsection{Linear Mechanics}
\begin{equation}\Sigma F_{ext} = \frac{\delta \vec{p}}{\delta t} = m\vec{a}_{com}\end{equation}
For constant acceleration:
\begin{equation}V_{f} = V_{i} + a\Delta t\end{equation}
\begin{equation}\Delta x = V_{i} \Delta t + \frac{1}{2}a\Delta t\end{equation}
\begin{equation}V_{f}^2 = V_{i}^2 + 2a\Delta x\end{equation}
Translational Kinetic Energy
\begin{equation}K_{t} = \frac{1}{2}mv^2\end{equation}
Linear Momentum
\begin{equation}p = m\vec{v}\end{equation}

\subsection{Rotational Mechanics}
\begin{equation}\Sigma \tau _{ext} = \frac{\delta \vec{L}}{\delta t} = I\vec{\alpha}\end{equation}
For constant angular acceleration:
\begin{equation}\omega_{f} = \omega_{i} + \alpha\Delta t\end{equation}
\begin{equation}\Delta \theta = \omega_{i} \Delta t + \frac{1}{2}\alpha\Delta t\end{equation}
\begin{equation}\omega_{f}^2 = \omega_{i}^2 + 2\alpha\Delta \theta\end{equation}
Angular Momentum
\begin{equation}\vec{L} = \vec{r} \times \vec{p}\end{equation}
Moment of Intertia
\begin{equation} I = \Sigma m_{i}r_{i}^2 = \int{r^2 dm}\end{equation}
Parallel Axis Theorem
\begin{equation}
  I_p = I_{cm} + Md^2
\end{equation}
Where M is the mass of the object, and d is the distance between the center of mass and the pivot point p. 

\subsection{Fluids}
\begin{equation}Pressure = \frac{Force}{Area}\end{equation}
Pressure at Depth
\begin{equation}P_{1} = P_{0} + \rho gh\end{equation}
Hydraulic Lift\\For same height\begin{equation} \frac{F_{1}}{A_{1}} = \frac{F_{2}}{A_{2}} \end{equation}
To lift $A_{2}$ by distance $d_{2}$, $F_{1}$ must be increased by
 \begin{equation} \Delta F_{1} = \rho g(A_{1} + A_{2})d_{2}\end{equation}
 Buouyancy
 \begin{equation}F_{B} = \rho_{fluid}V_{displaced}g\end{equation}
 Volumetric Flow Rate
  \begin{equation}Q = Area*Velocity\end{equation}
  Continuity Equation (Conservation of Flow)
  \begin{equation}\Sigma Q_{in} = \Sigma Q_{out} \end{equation}
  \begin{equation}Volume = Area*Velocity*\Delta t = Q\Delta t\end{equation}
Bernoulli's Equation
 \begin{equation}P_{1} + \frac{1}{2}\rho v_{i}^2 + \rho gy_{i} =P_{2} + \frac{1}{2}\rho v_{f}^2 + \rho gy_{f}\end{equation}

\subsection{Waves}
3D Wave Equation\begin{equation}\frac{1}{v^2} \frac{\partial^2\psi}{\partial t}= \nabla^2 \psi \end{equation}

\begin{equation}v=\sqrt{\frac{\tau}{\mu}},  \tau=tension, \mu=mass density \end{equation}

Traveling Waves\begin{equation}A\sin(kx+\omega t + \phi_0)+B\sin(kx+\omega t +\phi_0)\end{equation}

Amplitude = A\\\\
Phase Velocity\begin{equation}\vec{V}=\frac{\vec{\omega}}{k}\end{equation}
Group Velocity\begin{equation}\vec{V}=\frac{\delta \vec{\omega}}{\delta k}\end{equation}
Angular Frequency\begin{equation}\omega = 2\pi f \end{equation}
Wave vector\begin{equation}k = \frac{2\pi}{\lambda}\end{equation}
Period\begin{equation} T = \frac{2\pi}{\omega}\end{equation}

Standing Waves\begin{equation}A\sin(kx)\sin(\omega t+\phi_0)+B\cos(kx)\cos(\omega t+\phi_0)\end{equation}

Quality factor\begin{equation}Q=\frac{\omega}{2\beta}=2\pi \frac{energy-lost}{energy-stored}\end{equation}
$\Rightarrow$where $\beta$ is Full Width at Half Maximum

Beat Frequency\begin{equation}f_B=|f_1-f_2|.....T_B = \frac{1}{f_B}\end{equation}

Decibels(measured as Intensity {\bf above} threshold of human hearing at 1000 Hz)\begin{equation}I(dB)=10log_{10}\frac{I}{I_0}\end{equation}
$\Rightarrow$Where $I_0 = 10^{-12} \frac{watts}{m^2}$

Sonic Boom\begin{equation}\sin{\theta} = \frac{s}{v}\end{equation}
$\Rightarrow$ where s is the speed of sound, v speed of source.  **Note: the right angle when drawing this diagram is with respect to the wave front, not the path of the source.

Elastic Collisions\begin{equation}V_{f1}=V_{i1}(\frac{m_1-m_2}{m_1+m_2})+V_{i2}(\frac{2m_2}{m_1+m_2})\end{equation}
\begin{equation}V_{f2}=V_{i1}(\frac{2m_1}{m_1+m_2})+V_{i2}(\frac{m_2-m_1}{m_1+m_2})\end{equation}

Impulse\begin{equation}I = \int_{t_i}^{t_f} Fdt = \Delta p = J \end{equation}

{\bf{Impulse}}\begin{equation}I = \int_{t_i}^{t_f} Fdt = \Delta p = J \end{equation}

{\bf{Bragg's Law}}\begin{equation} m\lambda = 2d\sin{\theta}\end{equation}
$\Rightarrow$where m = 1,2,3...\\
\\
{\bf{Diffraction Grating}}\begin{equation}d(\sin{\theta}-\sin{\theta_i}) = m\lambda \end{equation}
\begin{equation}d = \frac{1}{N}\end{equation}
\hspace*{.5in}for normally incident:
\begin{equation}d\sin{\theta} = m\lambda\end{equation}
$\Rightarrow$ where m = 1,2,3... and N = number of slits per unit length\\
\\
{\bf{Pinhole Camera}}, optimal pinhole diameter:\begin{equation}\sqrt{f\lambda}\end{equation}
$\Rightarrow$ where f is the focal length NOT the frequency\\
\\
{\bf{Light Phase Shift at a boundary}}: Low index of refraction to high phase shift of $\pi$.  High to low no phase shift.

{\bf{Focal lengths}}\begin{equation}\frac{1}{S_o}+\frac{1}{S_i} = \frac{1}{f}\end{equation}
$\Rightarrow$ where $S_o$ is the distance to the object, $S_i$ the distance to the image.\\
\\
{\bf{Ohm's Law for AC Current}}\begin{equation}V = IZ\end{equation}
\begin{equation}Z = X + R\end{equation}
$\Rightarrow$ where X = reactance and R = resistance\\
\\
{\bf{Power dissapated - AC current}}\begin{equation}P = IV\cos{\phi}\end{equation}
$\Rightarrow$ where $\phi$ is the phase difference between the voltage and current 




\section{Quantum}
{\bf{Average values of bang gaps}}: Conductor ~1eV, Insulator ~10eV.\\
Commutators:
\begin{align}
  [A,B] &= AB - BA\\
  [AB,C] &= A[B,C] + [A,C]B
\end{align}






{\bf{Notes:}}\\
\hspace*{.5in}An $\emph{extrinsic}$ semi-conductor is one which has been n-type doped with electrons.\\
\hspace*{.5in}And $\emph{intrinsic}$ semi-conductor is one which has not been doped.




\section{E/M}
\{\bf{Maxwell's Equations in Differential form}} 
%mattson: there's some useful formatting stuff for vectors in latex
%incase you didn't know about them. here's an example:

%% \begin{equation}
%%   \vec{\nabla} \cdot \vec{E} = \frac{\rho(\vec{r})}{\epsilon_0}
%% \end{equation}

\begin{equation}\nabla \circ E=\frac{\rho}{\epsilon_0}\end{equation}
\begin{equation}\nabla \times B=\mu_0 J + \frac{dE}{dt}\end{equation}
\begin{equation}\nabla \times E=-\frac{dB}{dt}\end{equation}
\begin{equation}\nabla \circ B=0\end{equation}
$\Rightarrow$ Equations 3 and 4 are ``Sourcefull'' while 5 and 6 are ``Source free''\\
\\
{\bf{Maxwell's Equations in Integral form}}
\begin{equation}\int E \circ dA=\frac{q_{enc}}{\epsilon}\end{equation}
\begin{equation}\int B \circ dA=0\end{equation}
\begin{equation}\int E \circ dr=-\frac{\partial}{\partial t}\int B \circ dA \end{equation}
\begin{equation}\int B \circ dr=\mu_0 \int J \circ dA + \epsilon_0 \int \frac{dE}{dt}\circ dA\end{equation}

{\bf{Force on Charge particles}}\begin{equation}F=q(E+V\times B)\end{equation}

{\bf{Force density}}\begin{equation}F=\rho E + J\times V\end{equation}

{\bf{Field Energy density}}\begin{equation}=\frac{1}{2} (\epsilon_0 E^2 + \frac{1}{\mu_0 B^2})\end{equation}

{\bf{$\vec{P}$ density}} \begin{equation}\epsilon_0 E \times B = \epsilon_0 \mu_0 \vec{S}\end{equation}

{\bf{$\vec{L}$ density}}\begin{equation}\vec{r}\times (\epsilon_0 E \times B)\end{equation}

{\bf{E field from point charge}}\begin{equation}E=\frac{kq}{(r-r_0)^2}\end{equation}

{\bf{Potential from point charge}}\begin{equation}V=\frac{kq}{r-r_0}\end{equation}










\section{Thermal Physics} %and stat mech? 
{\bf Fermi-Dirac statistics}
\begin{equation}
  \dfrac{1}{\exp\left({\dfrac{\epsilon-\mu}{\tau}}\right)+1} < 1
\end{equation}
{\bf Bose-Einstein statistics}
\begin{equation}
  \dfrac{1}{\exp\left({\dfrac{\epsilon-\mu}{\tau}}\right)+1} \qquad \textrm{is unbounded}
\end{equation}
{\bf Photon}
\begin{equation}
  \dfrac{1}{\exp\left({\dfrac{\epsilon-\mu}{\tau}}\right)-1} \qquad \mu=0, \epsilon=\hbar\omega
\end{equation}
{\bf Maxwell-Boltzmann statistics}
\begin{equation}
  \exp{\left(-\frac{\epsilon-\mu}{\tau}\right)} < 1
\end{equation}

{\bf Bose-Einstein relations}
\begin{gather}
  D(\epsilon) \alpha \epsilon^{\frac{1}{2}} \\
  U \alpha T^{\frac{5}{2}}
  C_V \alpha T^{\frac{3}{2}}
\end{gather}
  
{\bf Equipartition Theorem}
\begin{equation}
  U = \frac{1}{2}Nk_{B}T \qquad \textrm{per degree of freedom}
\end{equation}

{\bf Van der Wals}
\begin{equation}
  \left(p + n^2\frac{a}{V^2}\right)\left(V-nb\right) = Nk_BT \qquad \textrm{n in moles}
\end{equation}

{\bf Ideal gas properties} %maybe explain later
\begin{gather}
  pV = Nk_bT \\
  U = \frac{3}{2}Nk_BT \\
  \frac{\partial U}{\partial T} = \frac{3}{2}Nk_B \\
  S = Nk_B\ln\left(\left(\frac{T}{T_0}\right)^{\frac{3}{2}}\frac{V}{V_0}\right) + S_0
\end{gather}

{\bf Density of states}
\begin{gather}
  \left(\frac{L}{2\pi\hbar}\right)^3\int 4\pi p^2 dp \qquad \textrm{or, with a change of variables} \\
  \left(\frac{L}{2\pi\hbar}\right)^3\int 4\pi \left(2m\right)^{\frac{3}{2}}\frac{\epsilon^{\frac{1}{2}}}{2} d\epsilon \qquad \textrm{since $L^3 = V$}
\end{gather}




\section{Special Relativity}









\section{Oscillations}
{\bf{Energy of an Oscillator}}\begin{equation}T+U=\frac{1}{2}mv^2 + \frac{1}{2}kx^2\end{equation}
{\bf{Physical Pendulum}}\begin{equation}\omega = \sqrt{\frac{g}{L}}.....T=2\pi \sqrt{\frac{L}{g}}\end{equation}

{\bf{Vibrations on a String:}}\\
\\
\hspace*{.5in}Fixed-Fixed\begin{equation}A\sin(kx)\cos(\omega t + \phi_0), k = \frac{n\pi}{L}\end{equation}

\hspace*{.5in}Fixed-Free\begin{equation}A\sin(kx)\cos(\omega t+\phi_0), k = \frac{(2n+1)\pi}{L}\end{equation}

\hspace*{.5in}Free-Free\begin{equation}A\sin(kx)\cos(\omega t + \phi_0), k= \frac{n\pi}{L}\end{equation}


\section{Wave Optics}










\section{Ray Optics}









\section{Laboratory and Electronics}
Polarized Light through Polarizer\begin{equation}I_{transmitted} = I_0\cos^2(\theta)\end{equation}

Non-Polarized Light through Polarizer\begin{equation}I_{transmitted}=\frac{1}{2}I_0\end{equation}

$\Rightarrow$ I still haven't the foggiest idea what this means








\section{Condensed Matter}









\section{Nuclear and Particle Physics}









\section{Mathematics and Statistics}
{\bf Noethers Theorem:} If a system has a continuous symmetry property, then there are corresponding quantities whose values are conserved in time.  These are as follows:

\begin{tabular}{ l | r}
  Symmetry & Invariant\\
  Time & Energy\\
  Translation & Linear Momentum\\
  Rotation & Angular Momentum\\
  Gauge? & Charge\\
  Reflection & Parity\\
\end{tabular}

${\bf General Fourier}$ \begin{equation}a_n=\int_0^T sin(\frac{n\pi}{L}x)f(x)dx\end{equation}
\begin{equation}f(x)=\sum_0^{\infty} a_n sin(n x)\end{equation}




\section{Cosmology}
Need to know something about Star Sequences








\section{Fluids}









\section{Constants}
Boltzmann's constant k=$1.38\times10^{-23} J/K$

e=$1.602\times10^{-11}C$

G=$6.672\times10^{-11}Nm^2/kg^2$

h=$6.626\times10^{-34}Js$

$N_A=6.022\times10^{23}molecules/mol$

Gas Constant(R)=8.3145 J/mol K

$m_e=9.109\times10^{-31}kg$

$m_n=1.672\times10^{-27}kg$

$m_p=1.672\times10^{-27}kg$

Permitivity of free space $\epsilon_0=8.854\times10^{-12}C^2/N m^2$

$\Rightarrow$ k=$\frac{1}{4\pi \epsilon_0}=8.987\times10^{9} Nm^2/C^2$

Permeability of free space $\mu_0=4\pi \times10^{-7}Nm^2/A m$

$\frac{1}{\epsilon_0 \mu_0} = c^2$









\section{Additional Questions/Review}







\end{document}

